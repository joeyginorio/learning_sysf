% For review submission
%\documentclass[acmsmall,review,anonymous]{acmart}
\documentclass[acmsmall]{acmart}

\settopmatter{printfolios=true,printccs=false,printacmref=false}
%% For double-blind review submission, w/ CCS and ACM Reference
%\documentclass[acmsmall,review,anonymous]{acmart}\settopmatter{printfolios=true}
%% For single-blind review submission, w/o CCS and ACM Reference (max submission space)
%\documentclass[acmsmall,review]{acmart}\settopmatter{printfolios=true,printccs=false,printacmref=false}
%% For single-blind review submission, w/ CCS and ACM Reference
%\documentclass[acmsmall,review]{acmart}\settopmatter{printfolios=true}
%% For final camera-ready submission, w/ required CCS and ACM Reference
%\documentclass[acmsmall]{acmart}\settopmatter{}


% Load any additional packages %%%%%%%%%%%%%%%%%%%%%%%%%%%%%%%%%%%%%%%%%%%%
\usepackage{amsmath,amsthm,amsfonts,wasysym,bussproofs}
\usepackage{csquotes}
\usepackage{booktabs,changepage}
\usepackage{lipsum}
\usepackage{setspace}

\usepackage{titlesec,epigraph, listings}
\usepackage{etoolbox}
\usepackage{enumitem}

% Commands %%%%%%%%%%%%%%%%%%%%%%%%%%%%%%%%%%%%%%%%%%%%%%%%%%%%%%%%%%%%%%%%
\newcommand{\lam}{\lambda}

\renewenvironment{proof}
    {\textit{Proof.}}
    {\qed\\}
    
\newtheoremstyle{mytheoremstyle} % name
    {}                    % Space above
    {}                    % Space below
    {\itshape}                   % Body font
    {}                 % Indent amount
    {\bfseries}                  % Theorem head font
    {.}                          % Punctuation after theorem head
    {\newline}                      % Space after theorem head
    {}                           % Theorem head spec (can be left empty, meaning ‘normal’)
\theoremstyle{mytheoremstyle}
\newtheorem{theorem}{Theorem}[section]
\newtheorem{corollary}[theorem]{Corollary}
\newtheorem{lemma}[theorem]{Lemma}




%% Journal information
%% Supplied to authors by publisher for camera-ready submission;
%% use defaults for review submission.
\acmJournal{PACMPL}
\acmVolume{1}
\acmNumber{CONF} % CONF = POPL or ICFP or OOPSLA
\acmArticle{1}
\acmYear{2018}
\acmMonth{1}
\acmDOI{} % \acmDOI{10.1145/nnnnnnn.nnnnnnn}
\startPage{1}

%% Copyright information
%% Supplied to authors (based on authors' rights management selection;
%% see authors.acm.org) by publisher for camera-ready submission;
%% use 'none' for review submission.
\setcopyright{none}
%\setcopyright{acmcopyright}
%\setcopyright{acmlicensed}
%\setcopyright{rightsretained}
%\copyrightyear{2018}           %% If different from \acmYear

%% Bibliography style
\bibliographystyle{ACM-Reference-Format}
%% Citation style
%% Note: author/year citations are required for papers published as an
%% issue of PACMPL.
\citestyle{acmauthoryear}   %% For author/year citations


%%%%%%%%%%%%%%%%%%%%%%%%%%%%%%%%%%%%%%%%%%%%%%%%%%%%%%%%%%%%%%%%%%%%%%
%% Note: Authors migrating a paper from PACMPL format to traditional
%% SIGPLAN proceedings format must update the '\documentclass' and
%% topmatter commands above; see 'acmart-sigplanproc-template.tex'.
%%%%%%%%%%%%%%%%%%%%%%%%%%%%%%%%%%%%%%%%%%%%%%%%%%%%%%%%%%%%%%%%%%%%%%


%% Some recommended packages.
\usepackage{booktabs}   %% For formal tables:
                        %% http://ctan.org/pkg/booktabs
\usepackage{subcaption} %% For complex figures with subfigures/subcaptions
                        %% http://ctan.org/pkg/subcaption


\begin{document}

%% Title information
\title{Learning in System F}

%% Author information
%% Contents and number of authors suppressed with 'anonymous'.
%% Each author should be introduced by \author, followed by
%% \authornote (optional), \orcid (optional), \affiliation, and
%% \email.
%% An author may have multiple affiliations and/or emails; repeat the
%% appropriate command.
%% Many elements are not rendered, but should be provided for metadata
%% extraction tools.

%% Author with single affiliation.
\author{Joey Velez-Ginorio}
\affiliation{
  \department{Brain \& Cognitive Sciences}
  \institution{Massachusetts Institute of Technology}
  \streetaddress{43 Vassar St.}
  \city{Cambridge}
  \state{Massachusetts}
  \postcode{02139}
  \country{U.S.A.}                    %% \country is recommended
}
\email{joeyv@mit.edu}   

%% Author with single affiliation.
\author{Nada Amin}
\affiliation{
  \department{John A. Paulson School of Engineering and Applied Sciences}
  \institution{Harvard University}
  \streetaddress{29 Oxford St.}
  \city{Cambridge}
  \state{Massachusetts}
  \postcode{02138}
  \country{U.S.A.}                    %% \country is recommended
}
\email{namin@seas.harvard.edu}


%% Abstract
%% Note: \begin{abstract}...\end{abstract} environment must come
%% before \maketitle command
\begin{abstract}
Program synthesis, type inhabitance, inductive programming, and theorem proving. Different names for the same problem: learning programs from data. Sometimes the programs are proofs, sometimes they’re terms. Sometimes data are examples, and sometimes they’re types. Yet the aim is the same. We want to construct a program which satisfies some data. We want to learn a program.

What might a programming language look like, if its programs could also be learned? We give it data, and it learns a program from it. This work shows that System F yields a simple approach for learning from types and examples. Beyond simplicity, System F gives us a guarantee on the soundness and completeness of learning. We learn correct programs, and can learn all observationally distinct programs in System F. Unlike previous works, we don't restrict what examples can be. As a result, we show how to learn arbitrary higher-order programs in System F from types and examples. 

\end{abstract}


%% Keywords
%% comma separated list
\keywords{Program Synthesis, Type Theory, Inductive Programming} 
\maketitle


\section{Introduction}
\subsection{A tricky learning problem}

Imagine we're teaching you a program. Your only data is the type $nat \!\to\! nat$. It takes a natural number, and returns a natural number. Any ideas? Perhaps a program which computes...
$$f(x) = x, \;\;\;\;\;\;f(x) = x + 1,\;\;\;\;\;\; f(x) = x + 2,\;\;\;\;\;\; f(x) = x + \cdots$$
The good news is that $f(x) = x + 1$ is correct. The bad news is that the data let you learn a slew of other programs too. It doesn't constrain learning enough if we want to teach $f(x) = x + 1$. As teachers, we can provide better data.

Round 2. Imagine we're teaching you a program. But this time we give you an example of the program's behavior. Your data are the type $nat \!\to\! nat$ and an example $f(1) = 2$. It takes a natural number, and seems to return its successor. Any ideas? Perhaps a program which computes...
$$f(x) = x + 1,\;\;\;\;\;\; f(x) = x + 2 - 1,\;\;\;\;\;\; f(x) = x + 3 - 2,\;\;\;\;\;\;\cdots$$
The good news is that $f(x) = x + 1$ is correct. And so are all the other programs, as long as we're agnostic to some details. Types and examples impose useful constraints on learning. It's the data we use when learning in System F \cite{girard1989proofs}.

Existing work can learn successor from similar data \cite{osera2015program, polikarpova2016program}. But suppose $nat$ is a church encoding. For some base type $A$, $nat \coloneqq (A \to A) \to (A \to A)$. Natural numbers are then higher-order functions. They take and return functions. In this context, existing work can no longer learn successor. 

\subsection{A way forward}

The difficulty is with how to handle functions in the return type. The type $nat \!\to\! nat$ returns a function, a program of type $nat$. To learn correct programs, you need to ensure candidates are the correct type or that they obey examples. Imagine we want to verify that our candidate program $f$ obeys $f(1)=2$. With the church encoding, $f(1)$ is a function, and so is $2$. To check $f(1)=2$ requires that we decide function equality---which is undecidable in a Turing-complete language \cite{sipser2006introduction}. Functions in the return type create this issue. There are two ways out.

\begin{enumerate}
\item Don't allow functions in the return type, keep Turing-completeness.
\item Allow functions in the return type, leave Turing-completeness.
\end{enumerate}

Route 1 is the approach of existing work. They don't allow functions in the return type, but keep an expressive Turing-complete language for learning. This can be a productive move, as many interesting programs don't return functions.

Route 2 is the approach we take. We don't impose restrictions on the types or examples we learn from. We instead sacrifice Turing-completeness. We choose a language where function equality is decidable, but still expressive enough to learn interesting programs. Our work shows that this too is a productive move, as many interesting programs return functions. This route leads us to several contributions:
\begin{itemize}
\item Detail how to learn arbitrary higher-order programs in System F. (Section 2 \& 3)
\item Prove the soundness and completeness of learning. (Section 2 \& 3)
\item Provide an implementation of learning, extending strong theoretical guarantees in practice. (Section 4 \& 5)
\end{itemize}


\section{System F}
We assume you are familiar with System F, the polymorphic lambda calculus. You should know its syntax, typing, and evaluation. If you don't, we co-opt its specification in \cite{pierce2002types}. For a comprehensive introduction we defer the confused or rusty there. Additionally, we provide the specification and relevant theorems in the appendix.

Our focus in this section is to motivate System F: its syntax, typing, and evaluation. And why properties of each are advantageous for learning. Treat this section as an answer to the following question: 
\begin{displayquote}
\textit{Why learn in System F?}
\centering
\end{displayquote}
\subsection{Syntax}

% minimal syntax
% expressive enough to encode the useful things
System F's syntax is simple. There aren't many syntactic forms. Whenever we state, prove, or implement things in System F we often use structural recursion on the syntax. A minimal syntax means we are succint when we state, prove, or implement those things.

While simple, the syntax is still expressive. We can encode many staples of typed functional programming: algebraic data types, inductive types, and more \cite{pierce2002types}. For example, consider this encoding of products:
\begin{align*}
\tau_1 \times \tau_2 \,&::=\; \forall \alpha.(\tau_1 \to \tau_2 \to \alpha) \to \alpha\\
\langle e_1,e_2 \rangle &::=\; \Lambda \alpha.\lambda f\!:\!(\tau_1 \to \tau_2 \to \alpha).fe_1e_2 
\end{align*}

\subsection{Typing}
% Typed programs are safe : progress + preservation. Later, we leverage this to ensure the safety of programs we learn.
System F is safe. Its typing ensures both progress and preservation, i.e. that well-typed programs do not get stuck and that they do not change type \cite{pierce2002types}. When we introduce learning, we lift this safety and extend it to programs we learn. Because we use this safety in later proofs, we state the progress and preservation theorems in the appendix.

\subsection{Evaluation}
% strongly normalizing, show example of functions equivalent
System F is strongly normalizing. All its programs terminate. As a result, we can use a simple procedure for deciding equality of programs (including functions). 
\begin{enumerate}
\item Run both programs until they terminate.
\item Check if they share the same normal form, up to alpha-equivalence (renaming of variables).
\item If they do, they are equal. Otherwise, unequal.
\end{enumerate}
For example, this decision procedure renders these programs equal:
$$\lambda x\!:\!\tau.x \;\;=_\beta\;\; (\lambda y\!:\!(\tau\to\tau).y)\lambda z\!:\!\tau.z$$
The decision procedure checks that two programs exist in the transitive reflexive closure of the evaluation relation. This only works because programs always terminate, a property we formally state in the appendix.
% Strong normalization means we can decide function equality through the fdollowwing procedure: evaluate term, compare up to alpha equivalence.

\section{Learning from Types}
% present the augmented relation
% state completeness/soundness result? or prove...

% dont need a figure, but state the normal form syntax like osera does
% then introduce the learning from types.




\begin{figure}[ht]
\centering
\setlength{\tabcolsep}{12pt}
\begin{tabular}{l r  l r}
\specialrule{.1em}{0em}{.2em}
\specialrule{.1em}{0em}{1em}
    \Large \textsc{Learning} & 
    &  & \framebox{$\Gamma \vdash \tau \rightsquigarrow e$}\\
    & & \\
    \multicolumn{2}{c}{
    \def\extraVskip{4pt}
    \def\labelSpacing{4pt}
    \def\defaultHypSeparation{\hskip .05in}
        \AxiomC{$x:\tau \in \Gamma$}
            \RightLabel{\textsc{(L-Var)}}
        \UnaryInfC{$\Gamma \vdash \tau \rightsquigarrow x$}
        \DisplayProof
    } &
    \multicolumn{2}{c}{
    \def\extraVskip{4pt}
    \def\labelSpacing{4pt}
    \def\defaultHypSeparation{\hskip .05in}
        \AxiomC{$\Gamma,\alpha \vdash \tau \rightsquigarrow e$}
            \RightLabel{\textsc{(G-TAbs)}}
        \UnaryInfC{$\Gamma \vdash \forall\alpha.\tau \rightsquigarrow \lam \alpha.e$}
        \DisplayProof
    }
    \\
    & &\\
    \multicolumn{2}{c}{
    \def\extraVskip{4pt}
    \def\labelSpacing{4pt}
    \def\defaultHypSeparation{\hskip .05in}
        \AxiomC{$\Gamma,x{:}\tau_1 \vdash \tau_2 \rightsquigarrow e_2$}
            \RightLabel{\textsc{(L-Abs)}}
        \UnaryInfC{$\Gamma \vdash \tau_1 \to \tau_2 \rightsquigarrow \lam x{:}\tau_1.e_2$}
        \DisplayProof
    } &
    \multicolumn{2}{c}{
    \def\extraVskip{4pt}
    \def\labelSpacing{4pt}
    \def\defaultHypSeparation{\hskip .05in}
        \AxiomC{$\Gamma \vdash \forall\alpha.\tau_1 \rightsquigarrow e$}
            \RightLabel{\textsc{(L-TApp)}}
        \UnaryInfC{$\Gamma \vdash [\tau_2/\alpha]\tau_1 \rightsquigarrow e\lceil\tau_2\rceil$}
        \DisplayProof
    }
    \\
    & &\\
    \multicolumn{2}{c}{
    \def\extraVskip{4pt}
    \def\labelSpacing{4pt}
    \def\defaultHypSeparation{\hskip .05in}
        \AxiomC{$\Gamma \vdash \tau_1 \to \tau_2 \rightsquigarrow e_1$}
        \AxiomC{$\Gamma \vdash \tau_1 \rightsquigarrow e_2$}
            \RightLabel{\textsc{(L-App)}}
        \BinaryInfC{$\Gamma \vdash \tau_2 \rightsquigarrow e_1e_2$}
        \DisplayProof
    } \\
    & \\
\specialrule{.1em}{1em}{0em}
\end{tabular}
\caption{Learning from types in System F}
    \label{fig:learning}
\end{figure}

\section{Learning from Examples}
% present the augmented relation
% state completeness/soundness result

\section{Implementation}

We have implemented a proof-of-concept prototype, with promising results.
(Reviewers: see the artifact attached.)

% how to make combinatorial search easier
% - enumerating normal form programs
% - deducing operationally distinct type applications
% - memoizing recurisve calls?
% - algebraic data types let us use examples productively
%    - sum types split examples
%	 - product types generate subproblems

\section{Experiments}

The implementation of learning from types is in the function {\tt genTerms} in {\tt learning.hs}. All the code in this section is in Haskell and runs in {\tt ghci}:
\begin{verbatim}
genTerms TyBool [] 5
\end{verbatim}

That generates all terms of type {\tt Bool} from the empty context, up to an AST size 5.

The implementation of learning from examples is in the function {\tt lrnTerms} in {\tt learning.hs}.
\begin{verbatim}
lrnTerms (TyAbs TyBool TyBool) [InTm TmTrue (Out TmTrue)] [] [] 3
\end{verbatim}

That generates all terms of type {\tt Bool->Bool} from the empty context, up to an AST size 3 \emph{and} which satisfy the example $<tt,tt>$. 

To generate polymorphic terms, our examples include types. These types are used to instantiate an example at a particular base type. For example, run the following in ghci to learn at type $(\forall X.X->X)$ with examples $<Bool,tt,tt>$:
\begin{verbatim}
lrnTerms (TyTAbs "X" (TyAbs (TyVar "X") (TyVar "X")))
         [InTy TyBool (InTm TmTrue (Out TmTrue))]
         [] [] 4
\end{verbatim}
This will produce the polymorphic identity function.
% TODO show the production

The implementation ``works'' minus a programs which require multiple type applications. There's also a bottleneck in performance that becomes apparent when synthesizing programs at around AST depth 20, because of the way the type application rule is currently implemented for learning.

\section{Related Work}

\subsection{Type-driven synthesis}

The seminal works (cite) of Osera et al. demonstrates that you can have none-``magical'' approaches to synthesis: everything is predictable, such as for instance, needing trace-complete examples (cite). We take inspiration from this work, and suggest a way forward for examples with higher-order functions in the input/output exammples.

Polikarpova et al. (cite) have extended the typed-driven approaach to refinement types. This is interesting because refinement types, such as Liquid Types (cite), are rather expressive and the types can act as rich specification. Our work extends the type-driven approach in an orthogonal direction.

\subsection{Other approaches to synthesis}

While the literature on synthesis is vast, we focus on three interesting alternatives to type-driven synthesis.

Program Sketching has been the most promising approach to synthesis. It relies on specification and holes.

Microsoft Prose (cite) has enabled the program demonstration of Excel. The Microsoft team has packaged the lessons learned from FlashFill (cite) into a generic framework Flash Meta (cite).

Another term for synthesis is program induction.
Metagol...
Neural-Guided Search uses neural networks \emph{not} for creating programs but for guiding the search on the symbolic possibilities.

\section{Conclusion}


%% Acknowledgments
%%\begin{acks}                            %% acks environment is optional
                                        %% contents suppressed with 'anonymous'
  %% Commands \grantsponsor{<sponsorID>}{<name>}{<url>} and
  %% \grantnum[<url>]{<sponsorID>}{<number>} should be used to
  %% acknowledge financial support and will be used by metadata
  %% extraction tools.
%%  This material is based upon work supported by the
%%  \grantsponsor{GS100000001}{National Science
%%    Foundation}{http://dx.doi.org/10.13039/100000001} under Grant
%%  No.~\grantnum{GS100000001}{nnnnnnn} and Grant
%%  No.~\grantnum{GS100000001}{mmmmmmm}.  Any opinions, findings, and
%%  conclusions or recommendations expressed in this material are those
%%  of the author and do not necessarily reflect the views of the
%%  National Science Foundation.
%%\end{acks}


%% Bibliography
\bibliography{refs.bib}

\newpage
%% Appendix
%%\appendix
\section{Appendix}
\subsection{Specification of System F}
\begin{figure}[h]
\centering
\setlength{\tabcolsep}{12pt}
\begin{tabular}{l  r}
\specialrule{.1em}{0em}{.2em}

\specialrule{.1em}{0em}{1em}
    \Large \textsc{Syntax} & \\
    & \\
    \begin{math}
    \setlength{\jot}{-2pt}
    \begin{aligned}
    e ::= \;& && \hspace*{.25in} \textsc{terms:}\\
        & x && \hspace*{.25in} \textit{variable}\\
        & e_1e_2 && \hspace*{.25in} \textit{application}\\
        & \lam x {:} \tau.e && \hspace*{.25in} \textit{abstraction}\\
        & e\lceil\tau\rceil && \hspace*{.25in} \textit{type application}\\    
        & \Lambda\alpha.e && \hspace*{.25in} \textit{type abstraction}\\
    \\
    v ::= \;& && \hspace*{.25in} \textsc{values:} \\
        & \lam x {:}\tau.e && \hspace*{.25in} \textit{abstraction}\\
        & \Lambda\alpha.e && \hspace*{.25in} \textit{type abstraction}\\
    \end{aligned}
    \end{math} & 
    \begin{math}
    \setlength{\jot}{-2pt}
    \begin{aligned}
    \tau ::= \;& && \hspace*{.25in} \textsc{types:}\\
        & \tau_1 \to \tau_2 && \hspace*{.25in} \textit{function type}\\
        & \forall\alpha.\tau && \hspace*{.25in} \textit{polymorphic type}\\
        & \alpha && \hspace*{.25in} \textit{type variable}\\
    \\
    \Gamma ::= \;& && \hspace*{.25in} \textsc{contexts:}\\
        & \cdot && \hspace*{.25in} \textit{empty}\\
        & x{:}\tau,\Gamma && \hspace*{.25in} \textit{variable}\\
        & \alpha,\Gamma && \hspace*{.25in} \textit{type variable}
    \end{aligned}
    \end{math}\\
    &\\
\specialrule{.1em}{1em}{0em}
\end{tabular}
\caption{Syntax in System F}
    \label{fig:syntax}
\end{figure}

\begin{figure}[h]
\centering
\setlength{\tabcolsep}{12pt}
\begin{tabular}{l r  l r}
\specialrule{.1em}{0em}{.2em}
\specialrule{.1em}{0em}{1em}
    \Large \textsc{Typing} & 
    &  & \framebox{$\Gamma \vdash e : \tau$}\\
    & & \\
    \multicolumn{2}{c}{
    \def\extraVskip{4pt}
    \def\labelSpacing{4pt}
    \def\defaultHypSeparation{\hskip .05in}
        \AxiomC{$x:\tau \in \Gamma$}
            \RightLabel{\textsc{(T-Var)}}
        \UnaryInfC{$\Gamma \vdash x : \tau$}
        \DisplayProof
    } &
    \multicolumn{2}{c}{
    \def\extraVskip{4pt}
    \def\labelSpacing{4pt}
    \def\defaultHypSeparation{\hskip .05in}
        \AxiomC{$\Gamma,\alpha \vdash e : \tau$}
            \RightLabel{\textsc{(T-TAbs)}}
        \UnaryInfC{$\Gamma \vdash \Lambda \alpha.e:\forall\alpha.\tau$}
        \DisplayProof
    }
    \\
    & &\\
    \multicolumn{2}{c}{
    \def\extraVskip{4pt}
    \def\labelSpacing{4pt}
    \def\defaultHypSeparation{\hskip .05in}
        \AxiomC{$\Gamma,x{:}\tau_1 \vdash e_2 : \tau_2$}
            \RightLabel{\textsc{(T-Abs)}}
        \UnaryInfC{$\Gamma \vdash \lam x{:}\tau_1.e_2:\tau_1 \to \tau_2$}
        \DisplayProof
    } &
    \multicolumn{2}{c}{
    \def\extraVskip{4pt}
    \def\labelSpacing{4pt}
    \def\defaultHypSeparation{\hskip .05in}
        \AxiomC{$\Gamma \vdash e : \forall\alpha.\tau_1$}
            \RightLabel{\textsc{(T-TApp)}}
        \UnaryInfC{$\Gamma \vdash e\lceil\tau_2\rceil : [\tau_2/\alpha]\tau_1$}
        \DisplayProof
    }
    \\
    & &\\
    \multicolumn{2}{c}{
    \def\extraVskip{4pt}
    \def\labelSpacing{4pt}
    \def\defaultHypSeparation{\hskip .05in}
        \AxiomC{$\Gamma \vdash e_1 : \tau_1 \to \tau_2$}
        \AxiomC{$\Gamma \vdash e_2 : \tau_1$}
            \RightLabel{\textsc{(T-App)}}
        \BinaryInfC{$\Gamma \vdash e_1e_2 : \tau_2$}
        \DisplayProof
    } \\
    & \\
\specialrule{.1em}{1em}{0em}
\end{tabular}
\caption{Typing in System F}
    \label{fig:typing}
\end{figure}

\begin{theorem}[\textsc{Progress in Typing}]
If $e$ is a closed, well-typed program, then either $e$ is a value or else there is some program $e'$ such that $e \to_\beta e'$.
\label{progress-typing}
\end{theorem}
\begin{theorem}[\textsc{Preservation in Typing}]
If $\,\Gamma \vdash e : \tau$ and $e \to_\beta e'$, then $\Gamma \vdash e' : \tau$.
\label{preservation-typing}
\end{theorem} 

\begin{figure}[h]
\centering
\setlength{\tabcolsep}{12pt}
\begin{tabular}{l r  l r}
\specialrule{.1em}{0em}{.2em}
\specialrule{.1em}{0em}{1em}
    \Large \textsc{Evaluating} & 
    &  & \fbox{ $e \to_\beta e'$}\\
    & & \\
    \multicolumn{2}{c}{
    \def\extraVskip{4pt}
    \def\labelSpacing{4pt}
    \def\defaultHypSeparation{\hskip .05in}
        \AxiomC{$e_1 \to_\beta e_1'$}
            \RightLabel{\textsc{(E-App1)}}
        \UnaryInfC{$e_1e_2 \to_\beta e_1'e_2$}
        \DisplayProof
    } &
    \multicolumn{2}{c}{
    \def\extraVskip{4pt}
    \def\labelSpacing{4pt}
    \def\defaultHypSeparation{\hskip .05in}
        \AxiomC{$e \to_\beta e'$}
            \RightLabel{\textsc{(E-TApp)}}
        \UnaryInfC{$e\lceil\tau\rceil \to_\beta e'\lceil\tau\rceil$}
        \DisplayProof
    }
    \\
    & &\\
    \multicolumn{2}{c}{
    \def\extraVskip{4pt}
    \def\labelSpacing{4pt}
    \def\defaultHypSeparation{\hskip .05in}
        \AxiomC{$e_2 \to_\beta e_2'$}
            \RightLabel{\textsc{(E-App2)}}
        \UnaryInfC{$e_1e_2 \to_\beta e_1e_2'$}
        \DisplayProof
    } &
    \multicolumn{2}{c}{
    \def\extraVskip{4pt}
    \def\labelSpacing{4pt}
    \def\defaultHypSeparation{\hskip .05in}
        \AxiomC{$(\Lambda\alpha.\lam x{:}\alpha.e)\lceil\tau\rceil \to_\beta (\lam x{:}\alpha.e)[\tau/\alpha]\,\,$\textsc{(E-TSub)}}
        \DisplayProof
    }
    \\
    & &\\
    \multicolumn{2}{c}{
    \def\extraVskip{4pt}
    \def\labelSpacing{4pt}
    \def\defaultHypSeparation{\hskip .05in}
        \AxiomC{$(\lam x{:}\tau.e)v \to_\beta e[v/x]\,\,$\textsc{(E-Sub)}}
        \DisplayProof
    } \\
    & \\
\specialrule{.1em}{1em}{0em}
\end{tabular}
\caption{Evaluating in System F}
    \label{fig:evaluating}
\end{figure}

\begin{theorem}[\textsc{Normalization in Evaluation}]
Well-typed programs in System F always evaluate to a value, to a normal form.
\label{normalization-evaluation}
\end{theorem}
%%Text of appendix \ldots

\end{document}
